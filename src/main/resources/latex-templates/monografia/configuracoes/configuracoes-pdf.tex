% CONFIGURAÇÕES DE APARÊNCIA DO PDF FINAL--------------------------------------
\definecolor{black}{RGB}{0,0,0}
\makeatletter
\hypersetup{%
    portuguese,
    colorlinks=true,   % true: "links" coloridos; false: "links" em caixas de texto
    linkcolor=black,    % Define cor dos "links" internos
	urlbordercolor=red,
    citecolor=black,    % Define cor dos "links" para as referências bibliográficas
    filecolor=blue,    % Define cor dos "links" para arquivos
    urlcolor=blue,     % Define a cor dos "hiperlinks"
    breaklinks=true,
    pdftitle={\@title},
    pdfauthor={\@author},
    pdfkeywords={abnt, latex, abntex, abntex2}
}
\makeatother

% ALTERA O ASPECTO DA COR AZUL--------------------------------------------------
\definecolor{blue}{RGB}{41,5,195}

% REDEFINIÇÃO DE LABELS---------------------------------------------------------
\renewcommand{\algorithmautorefname}{Algoritmo}
\def\equationautorefname~#1\null{Equa\c c\~ao~(#1)\null}

% CRIA ÍNDICE REMISSIVO---------------------------------------------------------
\makeindex

% HIFENIZAÇÃO DE PALAVRAS QUE NÃO ESTÃO NO DICIONÁRIO---------------------------
\hyphenation{%
    qua-dros-cha-ve
    Kat-sa-gge-los
}


\begin{comment}
    \lstdefinelanguage{JavaScript}{
      keywords={typeof, new, true, false, catch, function, return, null, catch, switch, var, if, in, while, do, else, case, break},
      %keywordstyle=\color{blue}\bfseries,
      ndkeywords={class, export, boolean, throw, implements, import, this},
      %ndkeywordstyle=\color{darkgray}\bfseries,
      %identifierstyle=\color{black},
      sensitive=false,
      comment=[l]{//},
      morecomment=[s]{/*}{*/},
      %commentstyle=\color{purple}\ttfamily,
      %stringstyle=\color{red}\ttfamily,
      morestring=[b]',
      morestring=[b]"
    }
\end{comment}

\lstdefinelanguage{texx}{
  keywords={section, subsection, chapter},
  %keywordstyle=\color{blue}\bfseries,
  ndkeywords={end, begin, documentclass},
  %ndkeywordstyle=\color{darkgray}\bfseries,
  %identifierstyle=\color{black},
  sensitive=false,
  comment=[l]{//},
  morecomment=[s]{/*}{*/},
  %commentstyle=\color{purple}\ttfamily,
  %stringstyle=\color{red}\ttfamily,
  morestring=[b]',
  morestring=[b]"
}
\lstset{ 
  backgroundcolor=\color{white},   % choose the background color; you must add \usepackage{color} or \usepackage{xcolor}; should come as last argument
  basicstyle=\footnotesize,        % the size of the fonts that are used for the code
  breakatwhitespace=false,         % sets if automatic breaks should only happen at whitespace
  breaklines=true,                 % sets automatic line breaking
  captionpos=b,                    % sets the caption-position to bottom
  %commentstyle=\color{mygreen},    % comment style
  deletekeywords={...},            % if you want to delete keywords from the given language
  escapeinside={\%*}{*)},          % if you want to add LaTeX within your code
  extendedchars=true,              % lets you use non-ASCII characters; for 8-bits encodings only, does not work with UTF-8
  firstnumber=1,                % start line enumeration with line 1000
  %frame=single,                       % adds a frame around the code
  keepspaces=true,                 % keeps spaces in text, useful for keeping indentation of code (possibly needs columns=flexible)
  %keywordstyle=\color{blue},       % keyword style
  language=Octave,                 % the language of the code
  morekeywords={*,...},            % if you want to add more keywords to the set
  numbers=left,                    % where to put the line-numbers; possible values are (none, left, right)
  numbersep=5pt,                   % how far the line-numbers are from the code
  numberstyle=\tiny\color{gray}, % the style that is used for the line-numbers
  rulecolor=\color{black},         % if not set, the frame-color may be changed on line-breaks within not-black text (e.g. comments (green here))
  showspaces=false,                % show spaces everywhere adding particular underscores; it overrides 'showstringspaces'
  showstringspaces=false,          % underline spaces within strings only
  showtabs=false,                  % show tabs within strings adding particular underscores
  stepnumber=1,                    % the step between two line-numbers. If it's 1, each line will be numbered
  %stringstyle=\color{mymauve},     % string literal style
  tabsize=4,                       % sets default tabsize to 2 spaces
  %title=\lstname                   % show the filename of files included with \lstinputlisting; also try caption instead of title
}