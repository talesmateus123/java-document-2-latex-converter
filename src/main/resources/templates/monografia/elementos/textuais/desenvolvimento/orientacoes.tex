% ORIENTAÇÕES GERAIS------------------------------------------------------------


% SOBRE AS ILUSTRAÇÕES----------------------------------------------------------
\chapter{Sobre as ilustrações}
\label{chap:apSobreIlust}

A seguir exemplifica-se como inserir ilustrações no corpo do trabalho. As ilustrações serão indexadas automaticamente em suas respectivas listas. A numeração sequencial de figuras, tabelas e equações também ocorre de modo automático.

Referências cruzadas são obtidas através dos comandos \verb|\label{}| e \verb|\ref{}|. Sendo assim, não é necessário por exemplo, saber que o número de certo capítulo é \ref{chap:fundamentacaoTeorica} para colocar o seu número no texto. Outra forma que pode ser utilizada é esta: \autoref{chap:fundamentacaoTeorica}, facilitando a inserção, remoção e manejo de elementos numerados no texto sem a necessidade de renumerar todos esses elementos.

% FIGURAS-----------------------------------------------------------------------
\chapter{FIGURAS}
\label{chap:figuras}

Exemplo de como inserir uma figura. A \autoref{fig:figura-exemplo1} aparece automaticamente na lista de figuras. Para saber mais sobre o uso de imagens no \LaTeX{} consulte literatura especializada \cite{Goossens2007}.

Os arquivos das figuras devem ser armazenados no diretório de "/dados".

\begin{figure}[!htb]
    \centering
    \caption{Exemplo de Figura}
    \includegraphics[width=0.5\textwidth]{./dados/figuras/figura1}
    \fonte{\citeonline{IRL2014}}
    \label{fig:figura-exemplo1}
\end{figure}

% QUADROS E TABELAS---------------------------------------------------------------
\chapter{QUADROS E TABELAS}
\label{chap:tabelas}

Exemplo de como inserir o \autoref{qua:quadro-exemplo1} e a \autoref{tab:tabela-exemplo1}. Ambos aparecem automaticamente nas suas respectivas listas. Para saber mais informações sobre a construção de tabelas no \LaTeX{} consulte literatura especializada \cite{Mittelbach2004}.

Ambos os elementos (Quadros e Tabelas) devem ser criados em arquivos separados para facilitar manutenção e armazenados no diretório de "/dados".

\begin{quadro}[!htb]
    \centering
    \caption{Exemplo de Quadro.\label{qua:quadro-exemplo1}}
    \begin{tabular}{|p{7cm}|p{7cm}|}
        \hline
        \textbf{BD Relacionais} & \textbf{BD Orientados a Objetos} \\
        \hline
        Os dados são passivos, ou seja, certas operações limitadas podem ser automaticamente acionadas quando os dados são usados. Os dados são ativos, ou seja, as solicitações fazem com que os objetos executem seus métodos. & Os processos que usam dados mudam constantemente. \\
        \hline
    \end{tabular}
    \fonte{\citeonline{Barbosa2004}}
\end{quadro}


A diferença entre quadro e tabela está no fato que um quadro é formado por linhas horizontais e verticais. Deve ser utilizado quando o conteúdo é majoritariamente não-numérico. O número do quadro e o título vem acima do quadro, e a fonte, deve vir abaixo. E Uma tabela é formada apenas por linhas verticais. Deve ser utilizada quando o conteúdo é majoritariamente numérico. O número da tabela e o título vem acima da tabela, e a fonte, deve vir abaixo, tal como no quadro.

\input{./dados/tabelas/tabela1}

% EQUAÇÕES-----------------------------------------------------------------------
\chapter{EQUAÇÕES}
\label{chap:equacoes}

Exemplo de como inserir a \autoref{eq:equacao-exemplo1} e a Eq. \ref{eq:equacao-exemplo2} no corpo do texto \footnote{Deve-se atentar ao fato de a formatação das equações ficar muito boa esteticamente.}. Observe que foram utilizadas duas formas distintas para referenciar as equações.

\begin{equation}
    X(s) = \int\limits_{t = -\infty}^{\infty} x(t) \, \text{e}^{-st} \, dt
    \label{eq:equacao-exemplo1}
\end{equation}

\begin{equation}
    F(u, v) = \sum_{m = 0}^{M - 1} \sum_{n = 0}^{N - 1} f(m, n) \exp \left[ -j 2 \pi \left( \frac{u m}{M} + \frac{v n}{N} \right) \right]
    \label{eq:equacao-exemplo2}
\end{equation}

% ALGORITMOS-----------------------------------------------------------------------
\chapter{ALGORITMOS}
\label{chap:algoritmos}

Exemplo de como inserir um algoritmo. Para inserção de algoritmos utiliza-se o pacote {\ttfamily algorithm2e} que já está devidamente configurado dentro do template.

Os algoritmos devem ser criados em arquivos separados para facilitar manutenção e armazenados no diretório de "/dados".\\
\\

\begin{algorithm}
    \caption{Exemplo de Algoritmo}
    \KwIn{o número $n$ de vértices a remover, grafo original $G(V, E)$}
    \KwOut{grafo reduzido $G'(V,E)$}
    $removidos \leftarrow 0$ \\
    \While {removidos $<$ n } {
        $v \leftarrow$ Random$(1, ..., k) \in V$ \\
            \For {$u \in adjacentes(v)$} {
                remove aresta (u, v)\\
                $removidos \leftarrow removidos + 1$\\
            }
            \If {há  componentes desconectados} {
                remove os componentes desconectados\\
            }
        }
\end{algorithm}


% SOBRE AS LISTAS--------------------------------------------------------------------
\chapter{SOBRE AS LISTAS}
\label{chap:apSobreLista}

Para construir listas de "\textit{bullets}"{} ou listas enumeradas, inclusive listas aninhadas, é utilizado o pacote \verb|paralist|.

Exemplo de duas listas não numeradas aninhadas, utilizando o comando \verb|\itemize|. Observe a indentação, bem como a mudança automática do tipo de "\textit{bullet}"{} nas listas aninhadas.

\begin{itemize}
    \item item não numerado 1
    \item item não numerado 2
    \begin{itemize}
        \item subitem não numerado 1
        \item subitem não numerado 2
        \item subitem não numerado 3
    \end{itemize}
    \item item não numerado 3
\end{itemize}

Exemplo de duas listas numeradas aninhadas, utilizando o comando \verb|\enumerate|. Observe a numeração progressiva e indentação das listas aninhadas.

\begin{enumerate}
    \item item numerado 1
    \item item numerado 2
    \begin{enumerate}
        \item subitem numerado 1
        \item subitem numerado 2
        \item subitem numerado 3
    \end{enumerate}
    \item item numerado 3
\end{enumerate}

% SOBRE AS CITAÇÕES E CHAMADAS DE REFERÊNCAS----------------------------------------------
\chapter{SOBRE AS CITAÇÕES E CHAMADAS DE REFERÊNCAS}
\label{chap:apSobreCita}

Citações são trechos de texto ou informações obtidas de materiais consultadss quando da elaboração do trabalho. São utilizadas no texto com o propósito de esclarecer, completar e embasar as ideias do autor. Todas as publicações consultadas e utilizadas (por meio de citações) devem ser listadas, obrigatoriamente, nas referências bibliográficas, para preservar os direitos autorais. São classificadas em citações indiretas e diretas.

% CITAÇÕES INDIRETAS-----------------------------------------------------------------------
\chapter{CITAÇÕES INDIRETAS}
\label{chap:citacoesLivres}

É a transcrição, com suas próprias palavras, das idéias de um autor, mantendo-se o sentido original. A citação indireta é a maneira que o pesquisador tem de ler, compreender e gerar conhecimento a partir do conhecimento de outros autores. Quanto à chamada da referência, ela pode ser feita de duas maneiras distintas, conforme o nome do(s) autor(es) façam parte do seu texto ou não. Exemplo de chamada fazendo parte do texto:\\
\\Enquanto \citeonline{Maturana2003} defendem uma epistemologia baseada na biologia. Para os autores, é necessário rever \ldots.\\

A chamada de referência foi feita com o comando \verb|\citeonline{chave}|, que produzirá a formatação correta.

A segunda forma de fazer uma chamada de referência deve ser utilizada quando se quer evitar uma interrupção na sequência do texto, o que poderia, eventualmente, prejudicar a leitura. Assim, a citação é feita e imediatamente após a obra referenciada deve ser colocada entre parênteses. Porém, neste caso específico, o nome do autor deve vir em caixa alta, seguido do ano da publicação. Exemplo de chamada não fazendo parte do texto:\\
\\Há defensores da epistemologia baseada na biologia que argumentam em favor da necessidade de \ldots \cite{Maturana2003}.\\

Nesse caso a chamada de referência deve ser feita com o comando \verb|\cite{chave}|, que produzirá a formatação correta.

% CITAÇÕES DIRETAS-----------------------------------------------------------------------
\chapter{CITAÇÕES DIRETAS}
\label{chap:citacoesLiterais}

É a transcrição ou cópia de um parágrafo, de uma frase, de parte dela ou de uma expressão, usando exatamente as mesmas palavras adotadas pelo autor do trabalho consultado.

Quanto à chamada da referência, ela pode ser feita de qualquer das duas maneiras já mencionadas nas citações indiretas, conforme o nome do(s) autor(es) façam parte do texto ou não. Há duas maneiras distintas de se fazer uma citação direta, conforme o trecho citado seja longo ou curto.

Quando o trecho citado é longo (4 ou mais linhas) deve-se usar um parágrafo específico para a citação, na forma de um texto recuado (4 cm da margem esquerda), com tamanho de letra menor e espaçamento entrelinhas simples. Exemplo de citação longa:
\\\begin{citacao}
    Desse modo, opera-se uma ruptura decisiva entre a reflexividade filosófica, isto é a possibilidade do sujeito de pensar e de refletir, e a objetividade científica. Encontramo-nos num ponto em que o conhecimento científico está sem consciência. Sem consciência moral, sem consciência reflexiva e também subjetiva. Cada vez mais o desenvolvimento extraordinário do conhecimento científico vai tornar menos praticável a própria possibilidade de reflexão do sujeito sobre a sua pesquisa \cite[p.~28]{Silva2000}.
\end{citacao}

Para fazer a citação longa deve-se utilizar os seguintes comandos:
\begin{verbatim}
\begin{citacao}
<texto da citacao>
\end{citacao}
\end{verbatim}

No exemplo acima, para a chamada da referência o comando \verb|\cite[p.~28]{Silva2000}| foi utilizado, visto que os nomes dos autores não são parte do trecho citado. É necessário também indicar o número da página da obra citada que contém o trecho citado.

Quando o trecho citado é curto (3 ou menos linhas) ele deve inserido diretamente no texto entre aspas. Exemplos de citação curta:\\
\\A epistemologia baseada na biologia parte do princípio de que "assumo que não posso fazer referência a entidades independentes de mim para construir meu explicar" \cite[p.~35]{Maturana2003}.\\
\\A epistemologia baseada na biologia de \citeonline[p.~35]{Maturana2003} parte do princípio de que "assumo que não posso fazer referência a entidades independentes de mim para construir meu explicar".

% DETALHES SOBRE AS CHAMADAS DE REFERÊNCIAS---------------------------------------------------------
\chapter{DETALHES SOBRE AS CHAMADAS DE REFERÊNCIAS}
\label{chap:referUtilizadas}

Outros exemplos de comandos para as chamadas de referências e o resultado produzido por estes:\\
\\\citeonline{Maturana2003} \ \ \  \verb|\citeonline{Maturana2003}|\\
\citeonline{Barbosa2004} \ \ \   \verb|\citeonline{Barbosa2004}|\\
\cite[p.~28]{Silva2000} \ \ \  \verb|\cite[p.~28]{Silva2000}|\\
\citeonline[p.~33]{Silva2000} \ \ \   \verb|\citeonline[p.~33]{v}|\\
\cite[p.~35]{Maturana2003} \ \ \   \verb|\cite[p.~35]{Maturana2003}|\\
\citeonline[p.~35]{Maturana2003} \ \ \   \verb|\citeonline[p.~35]{Maturana2003}|\\
\cite{Barbosa2004,Maturana2003} \ \ \   \verb|\cite{Barbosa2004,Maturana2003}|\\

% SOBRE AS REFERÊNCIAS BIBLIOGRÁFICAS-------------------------------------------------------
\chapter{SOBRE AS REFERÊNCIAS BIBLIOGRÁFICAS}
\label{chap:apSobreRefer}

A bibliografia é feita no padrão \textsc{Bib}\TeX{}. As referências são colocadas em um arquivo separado. Neste template as referências são armazenadas no arquivo "base-referencias.bib".

Existem diversas categorias documentos e materiais componentes da bibliografia. A classe abn\TeX{} define as seguintes categorias (entradas):

\begin{verbatim}
@book
@inbook
@article
@phdthesis
@mastersthesis
@monography
@techreport
@manual
@proceedings
@inproceedings
@journalpart
@booklet
@patent
@unpublished
@misc
\end{verbatim}

Cada categoria (entrada) é formatada pelo pacote \citeonline{abnTeX22014d} de uma forma específica. Algumas entradas foram introduzidas especificamente para atender à norma \citeonline{NBR6023:2002}, são elas: \verb|@monography|, \verb|@journalpart|,\verb|@patent|. As demais entradas são padrão \textsc{Bib}\TeX{}. Para maiores detalhes, refira-se a \citeonline{abnTeX22014d}, \citeonline{abnTeX22014b}, \citeonline{abnTeX22014c}.

% NOTAS DE RODAPÉ--------------------------------------------------------------------------
\chapter{NOTAS DE RODAPÉ}
\label{chap:notasRodape}

As notas de rodapé pode ser classificadas em duas categorias: notas explicativas\footnote{é o tipo mais comum de notas que destacam, explicam e/ou complementam o que foi dito no corpo do texto, como esta nota de rodapé, por exemplo.} e notas de referências. A notas de referências, como o próprio nome ja indica, são utilizadas para colocar referências e/ou chamadas de referências sob certas condições.

