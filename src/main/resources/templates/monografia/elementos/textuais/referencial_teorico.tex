% ----------------------------------------------------------
%Nesta parte, explica-se a teoria que está sendo estudada (o em basamento) e cita-se estudos anteriores sobre o ramo, ou semelhantes

%Não escreve nada sem autoria, tudo é embasado em um autor
%Tudo deve ser referenciado

%Ao excrever, fazer uma narrativa histórica, finalizar esta parte sistematizando os estudos, evidenciando as críticas encontradas na literatura, e as lacunas a serem preenchidas

%É no final dessa parte que se estabelece se haverão futuros questionários ou roteiros

%Que estudos citar? 
%1 - Journals (internacionais/nacionais)
%2 - Livros
%3 - Teses/dissertaçoes
%4 - Anais de eventos científicos
%5 - Base de dados oficiais

%O que não sao recomendados?
%1 - Textos não acadêmicos (virtuais ou impressos)
%2 - Reportagens e textos jornalísticos
%3 - Documentos sem identificação de fonte

%\textit{FRONT-END}	Parte responsável do desenvolvimento de \textit{software} por interagir com a entrada do usuário, para que o \textit{back-end} possa realizar o processamento
%\textit{BACK-END}	Parte responsável do desenvolvimento de \textit{software} por receber a entrada do usuário, a fim de realizar processamento e solicitações%Nesta parte, explica-se a teoria que está sendo estudada (o em basamento) e cita-se estudos anteriores sobre o ramo, ou semelhantes

%Não escreve nada sem autoria, tudo é embasado em um autor
%Tudo deve ser referenciado

%Ao excrever, fazer uma narrativa histórica, finalizar esta parte sistematizando os estudos, evidenciando as críticas encontradas na literatura, e as lacunas a serem preenchidas

%É no final dessa parte que se estabelece se haverão futuros questionários ou roteiros

%Que estudos citar? 
%1 - Journals (internacionais/nacionais)
%2 - Livros
%3 - Teses/dissertaçoes
%4 - Anais de eventos científicos
%5 - Base de dados oficiais

%O que não sao recomendados?
%1 - Textos não acadêmicos (virtuais ou impressos)
%2 - Reportagens e textos jornalísticos
%3 - Documentos sem identificação de fonte

%\textit{FRONT-END}	Parte responsável do desenvolvimento de \textit{software} por interagir com a entrada do usuário, para que o \textit{back-end} possa realizar o processamento
%\textit{BACK-END}	Parte responsável do desenvolvimento de \textit{software} por receber a entrada do usuário, a fim de realizar processamento e solicitações
% ----------------------------------------------------------

\chapter{Referencial teórico}
	\label{chap:referencial_teorico}
	{referencialTeorico}